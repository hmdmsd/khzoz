\chapter*{Résumé}
%\addcontentsline{toc}{chapter}{Résumé}

\lettrine{\textbf{C}}\lowercase{e} rapport présente les travaux réalisés chez Sopra Steria dans le cadre du stage de fin d'études, portant sur la conception et l'implémentation d'un framework d'automatisation des tests pour le projet SNCF COTS (Catalogue des Offres de Transport et de Services). Cette mission d'ingénierie complexe vise à transformer radicalement les processus d'assurance qualité d'un système critique par l'automatisation intelligente des tests de régression.

Le défi technique principal consiste à concevoir une solution d'automatisation native pour une architecture microservices distribuée, intégrant les principes du Behavior-Driven Development (BDD) via Cucumber avec l'écosystème Spring Boot existant. Cette innovation méthodologique résout la problématique de validation exhaustive de neuf services critiques tout en maintenant l'expressivité métier nécessaire à la collaboration avec les analystes fonctionnels.

L'approche d'ingénieur adoptée s'articule autour d'une méthodologie rigoureuse comprenant l'analyse systémique de l'architecture existante, la conception d'une solution modulaire et évolutive, l'implémentation de mécanismes de robustesse avancés et l'intégration transparente dans le pipeline CI/CD Jenkins. Cette démarche assure une transformation contrôlée des processus avec validation continue des performances et de la fiabilité.

Les résultats quantitatifs confirment l'efficacité de la solution développée : réduction de 85% des temps de validation (de 8 heures à 1h15), amélioration de la fiabilité à 94% avec moins de 2% de faux positifs, et augmentation de 40% de la couverture fonctionnelle. Ces performances valident l'approche technique et démontrent la création de valeur opérationnelle significative.

L'analyse multidimensionnelle révèle des impacts dépassant largement le cadre technique initial. La dimension organisationnelle illustre la transformation des pratiques de travail et l'évolution des rôles professionnels. La dimension environnementale et sociétale confirme la contribution à la durabilité numérique et à l'amélioration du service public ferroviaire. La dimension stratégique et économique démontre un retour sur investissement de 350% sur cinq ans et une différenciation concurrentielle pour Sopra Steria.

Cette mission valide l'acquisition des compétences d'ingénieur selon le référentiel IMT Atlantique, démontrant la capacité d'analyse de problématiques complexes, de conception de solutions innovantes, de prise de décision éclairée et d'intégration des enjeux sociétaux dans l'exercice du métier d'ingénieur.

\\[0.3cm]
\\
\rule{\linewidth}{0.2mm} \\[0.2cm]
\textbf {Mots-clés :} automatisation des tests, framework Cucumber, architecture microservices, BDD, CI/CD, Jenkins, Spring Boot, assurance qualité, transformation numérique, SNCF COTS, ingénierie logicielle, DevOps. \\
\rule{\linewidth}{0.2mm}