% ========================================================================
% INTRODUCTION GÉNÉRALE MISE À JOUR
% ========================================================================

\chapter*{Introduction générale}
\addcontentsline{toc}{chapter}{Introduction générale}

\lettrine{\textbf{L}}\lowercase{a} transformation numérique des systèmes critiques constitue un défi majeur pour les organisations modernes, particulièrement dans le secteur ferroviaire où la fiabilité, la performance et la sécurité sont impératives. Dans ce contexte, la Société Nationale des Chemins de fer Français (SNCF) s'engage dans une modernisation ambitieuse de son système d'information à travers le projet COTS (Catalogue des Offres de Transport et de Services). Cette initiative stratégique, menée en partenariat avec Sopra Steria, vise à créer un référentiel unifié remplaçant les multiples systèmes patrimoniaux disparates, impactant directement l'expérience de millions de voyageurs quotidiennement et les processus opérationnels critiques du transport ferroviaire français.

Face à la complexité croissante de l'écosystème COTS, composé de neuf microservices critiques interconnectés dans une architecture distribuée moderne, l'assurance qualité représente un enjeu stratégique et technique majeur. Les processus actuels de tests de non-régression, réalisés manuellement via l'outil ALM, génèrent des cycles de validation de huit heures et mobilisent intensivement les équipes qualité sur des tâches répétitives à faible valeur ajoutée. Cette problématique s'intensifie avec l'accélération des cycles de développement agiles, la nécessité d'intégrer des pratiques de livraison continue et l'impératif de maintenir des standards de qualité rigoureux pour un système ferroviaire où toute défaillance peut avoir des conséquences opérationnelles critiques.

\textbf{L'enjeu principal de cette mission de stage} consiste donc à concevoir et implémenter une solution d'automatisation complète et innovante des tests de régression pour le projet COTS. Cette mission d'ingénieur s'articule autour de la création d'un framework de tests automatisés utilisant l'approche Behavior-Driven Development (BDD) avec Cucumber, intégré nativement à l'architecture Spring Boot existante, et l'orchestration de cette solution dans le pipeline CI/CD Jenkins. L'objectif est de transformer fondamentalement les pratiques d'assurance qualité en réduisant drastiquement les temps d'exécution des campagnes de test (objectif de réduction de 80\%) tout en améliorant significativement leur fiabilité, leur reproductibilité et leur expressivité métier.

Cette problématique technique complexe nécessite une approche d'ingénieur rigoureuse et multidimensionnelle. Sur le plan technique et scientifique, elle implique la maîtrise des frameworks de test modernes, la conception d'architectures d'automatisation robustes, l'implémentation de mécanismes de gestion de la complexité et l'intégration harmonieuse dans des environnements de production critiques. Sur le plan organisationnel, elle requiert une compréhension approfondie des impacts sur les équipes de développement et d'assurance qualité, des processus de conduite du changement et de l'évolution des pratiques professionnelles. Les dimensions environnementale et sociétale touchent à l'optimisation des ressources informatiques, la réduction de l'empreinte carbone numérique et l'amélioration de la qualité des services ferroviaires contribuant au transport durable. Enfin, l'analyse stratégique et économique porte sur la valeur ajoutée pour Sopra Steria, l'impact économique pour la SNCF et les enjeux concurrentiels du secteur ferroviaire en transformation.

Le choix de ce stage s'inscrit pleinement dans mon projet professionnel d'ingénieur spécialisé en développement logiciel, DevOps et ingénierie de la qualité. Cette mission offre l'opportunité unique de travailler sur un projet d'innovation technique à fort impact sociétal, tout en développant une expertise approfondie des systèmes critiques, des architectures microservices et des pratiques d'assurance qualité modernes. L'environnement technologique avancé de Sopra Steria, la complexité du projet COTS et la collaboration avec des experts métier ferroviaires constituent un cadre d'apprentissage exceptionnel pour acquérir les compétences d'ingénieur recherchées dans le secteur de la transformation numérique et développer une vision systémique des enjeux de l'ingénierie logicielle contemporaine.

Ce rapport structure l'analyse de cette mission d'ingénieur en quatre chapitres progressifs et complémentaires. Le \textbf{premier chapitre} présente le contexte professionnel chez Sopra Steria et détaille le projet COTS dans son environnement technique, stratégique et organisationnel, établissant les fondements nécessaires à la compréhension des enjeux multidimensionnels. Le \textbf{deuxième chapitre} expose la conception et la réalisation du framework d'automatisation, depuis l'analyse systémique des problématiques jusqu'à l'implémentation complète des solutions techniques et leur intégration CI/CD, incluant l'évaluation critique des résultats obtenus. Le \textbf{troisième chapitre} développe l'analyse multidimensionnelle requise pour une formation d'ingénieur, examinant successivement les dimensions technique et scientifique, organisationnelle, environnementale et sociétale, ainsi que stratégique et économique du projet, révélant les impacts complexes de cette innovation technique. Le \textbf{chapitre final} propose une analyse critique approfondie de l'expérience d'ingénieur, évaluant la pertinence des approches méthodologiques employées, les compétences développées et transférables, et offrant des perspectives d'évolution professionnelle ainsi que des recommandations stratégiques pour l'entreprise et le secteur.